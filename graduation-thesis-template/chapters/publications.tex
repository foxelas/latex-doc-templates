\pagebreak
\textbf{{\huge Publications}}
\addcontentsline{toc}{chapter}{Publications}

\vfill 

\textbf{{\large Published Papers}}
\begin{description}
\item[Conference Paper] A, B, and C , ``we did this ," in IEIRER 2019 , city Japan, 24-26 Nov. 2019.
\end{description}

\vfill 

\textbf{{\large Scheduled}} 
\begin{description}
\item[Conference Paper] , B, and C , ``we did this ," in IEIRER 2019, city Japan, 24-26 Nov. 2019.
\end{description}

\vfill 

\textbf{{\large Development Codes}} \\
\begin{description}
\item[GitHub] The processing procedure of macroMSI was implemented using MATLAB version R2019a (Mathworks Inc.) and Python version 3.7. Specifically, 
MATLAB was used for image import, preprocessing, feature extraction, segmentation and graph production. Python was used for dimension reduction, training, validation and testing of malignancy classification. The entire developed code is available on GitHub at \href{hhttps://foxelas.github.io}{\link{foxelas.github.io}}. The code is eligible for free use under the MIT license.
\end{description}


\vfill 
