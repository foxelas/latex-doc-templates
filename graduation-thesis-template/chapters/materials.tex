\chapter{Implementation and Evaluation}

\section{Captured Dataset}


\begin{figure}[H]
\begin{center}
  \begin{subfigure}[t]{.3\linewidth}
    \centering\includegraphics[width=.9\linewidth]{example-image-a}
    \caption{}
  \end{subfigure}
    \begin{subfigure}[t]{.3\linewidth}
    \centering\includegraphics[width=.9\linewidth]{example-image-a}
    \caption{}
  \end{subfigure}
    \begin{subfigure}[t]{.3\linewidth}
    \centering\includegraphics[width=.9\linewidth]{example-image-a}
    \caption{}
  \end{subfigure}
  ~
  \\
  \begin{subfigure}[t]{.3\linewidth}
    \centering\includegraphics[width=.9\linewidth]{example-image-a}
    \caption{}
  \end{subfigure}
    \begin{subfigure}[t]{.3\linewidth}
    \centering\includegraphics[width=.9\linewidth]{example-image-a}
    \caption{}
  \end{subfigure}
    \begin{subfigure}[t]{.3\linewidth}
    \centering\includegraphics[width=.9\linewidth]{example-image-a}
    \caption{}
  \end{subfigure}
  ~
  \\
  \begin{subfigure}[t]{.4\linewidth}
    \centering\includegraphics[width=.9\linewidth]{example-image-a}
    \caption{}
  \end{subfigure}
    \begin{subfigure}[t]{.4\linewidth}
    \centering\includegraphics[width=.9\linewidth]{example-image-a}
    \caption{}
  \end{subfigure}
  ~
  \\
  \begin{subfigure}[t]{.4\linewidth}
    \centering\includegraphics[width=.9\linewidth]{example-image-a}
    \caption{}
  \end{subfigure}
    \begin{subfigure}[t]{.4\linewidth}
    \centering\includegraphics[width=.9\linewidth]{example-image-a}
    \caption{}
  \end{subfigure}
  
    \caption{The 10 specimens that comprise the database. These RGB  images were captured according to the standard procedure.}
  \label{fig:data}
\end{center}
\end{figure}

\section{Capturing Procedure}
\blindtext 

\section{Segmentation}
\blindtext 

\section{Estimation}
\blindtext 


\subsection{Simple Estimation}
\noindent If we assume that the product $S_{i}(\lambda)I_{i}(\lambda)$ is known, then reflectance reconstruction from MSI values can be regarded as a linear inverse problem solved by Wiener estimation. 

\par Formula \ref{eq:fullmat} can be re-written in the discrete space in a matrix form as:
\begin{equation}
	\mathbf{g} = \mathbf{Hr} + \mathbf{n},
	\label{eq:fullmat}
\end{equation}
where:
\begin{conditions}
 $g$ &  N-dimensional vector of intensities at N channels \\
 $r$ &  M-dimensional vector of reflectance at M wavelengths \\   
 $n$ &  N-dimensional vector of noise at N channels \\
 $H$ &  the product $S \cdot I$ NxM matrix of the system parameters\\
\end{conditions}


\begin{align}\label{eq:proof}
	\langle \mathbf{r}^{T}\mathbf{A}\mathbf{g} \rangle &= \langle \sum_{i=1}^{M} \sum_{j=1}^{N} r_{j}a_{ij}g_{i} \rangle \nonumber \\
	&= \langle \mathbf{A} \mathbf{g}^{T}\mathbf{r} \rangle \nonumber \\
	&= \langle \mathbf{A} \rangle \langle \mathbf{g}^{T}\mathbf{r}\rangle \nonumber \\
	&= \mathbf{A} \langle \mathbf{g}^{T}\mathbf{r}\rangle
\end{align}


\subsection{Filter}
\blindtext

\begin{align}
&\mathbf{\mathcal{R}} = [\mathbf{r}_{cent-(M^2-1)/2}, ..., \mathbf{r_{cent}, ..., \mathbf{r}_{cent+(M^2-1)/2}}]^T  \nonumber \\
&\mathbf{\mathcal{G}} = [\mathbf{g}_{cent-(M^2-1)/2}, ..., \mathbf{g_{cent}, ..., \mathbf{g}_{cent+(M^2-1)/2}}]^T \nonumber  \\
&\mathbf{\mathcal{N}} = [\mathbf{n}_{cent-(M^2-1)/2}, ..., \mathbf{n_{cent}, ..., \mathbf{n}_{cent+(M^2-1)/2}}]^T
\end{align}

\noindent Now, using the Kronecker product, the spatio-spectral system response from eq. \ref{eq:proof} can be rewritten as:

\begin{equation}
C = \langle  \mathbf{r}_{cent} \mathcal{R}^{T} \rangle  (\mathbf{I} \otimes \mathbf{H})^T [(\mathbf{I} \otimes \mathbf{H})  \langle \mathbf{\mathcal{R}}\mathbf{\mathcal{R}}^{T} \rangle (\mathbf{I} \otimes \mathbf{H})^{T} +  \langle \mathbf{\mathcal{N}} \mathbf{\mathcal{N}}^{T} \rangle]^{-1}
\end{equation}


\subsection{Comparison of Reconstruction Methods}
\blindtext

\section{Texture description}
\blindtext


\section{Dimension Reduction}
\blindtext

