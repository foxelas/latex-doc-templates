\chapter{Fundamentals}

\section{Procedure}

\subsection{Current Procedure}

\par \blindtext 

\begin{comment}
this is hidden 
\end{comment}

\begin{figure}[ht]
\begin{center}
  \includegraphics[width=90mm]{example-image-a}
  \caption{The current imaging process during pathology. The camera symbol denotes the occurrence of image capturing at the specific state of the tissue specimen.}
  \label{fig:currentimaging}
\end{center}
\end{figure}



%\begin{landscape}
%\begin{sidewaysfigure}
\begin{figure}[H]
\begin{center}
  \includegraphics[width=\linewidth]{example-image-a}
  \caption{Work-flow of the current procedure of skin cancer diagnosis}
  \label{fig:currentdiagnosis}
\end{center}
\end{figure}
%\end{sidewaysfigure}
%\end{landscape}

\par The usual work-flow of the process is described in Figure \ref{fig:currentdiagnosis}. \Blindtext  



\begin{figure}[ht]%[hbt!]
\begin{center}

  %\begin{subfigure}[t]{.45\linewidth}
   % \centering\includegraphics[width=.9\linewidth]{diag2.jpg}
   % \caption{}
  %\end{subfigure}
  \begin{subfigure}[t]{.45\linewidth}
    \centering\includegraphics[width=.9\linewidth]{example-image-a}
    \caption{}
  \end{subfigure}
 % ~
 % \\
 % \begin{subfigure}[t]{.45\linewidth}
  %  \centering\includegraphics[width=.9\linewidth]{cutdiag1.jpg}
 %   \caption{}
 % \end{subfigure}
   \begin{subfigure}[t]{.45\linewidth}
    \centering\includegraphics[width=.9\linewidth]{example-image-a}
    \caption{}
  \end{subfigure}
  
  \caption{Example of a captured in current procedure, showing sections (a) before and (b) after.}
  \label{fig:diagnosis}
\end{center}
\end{figure}

\subsection{Shortcomings}
\label{sec:humanperf}
\par \blindtext



\begin{table}[H]
\centering
\caption{Performance of detection.}
\label{tabl:humanperf}
	\begin{tabular}{p{3cm} p{2cm} p{5cm} p{1cm} p{2cm}} 
 		\toprule
 		\textbf{Study} & \textbf{Total Cases} & \textbf{Disease} & 
 		\textbf{PPV} & \textbf{Sensitivity} \\
 		\midrule
 		\textbf{A et al.}  & 8694 &  C & 49.4\% & 41.1\% \\
 		 (2008) & & CC & 72.7\% & 63.9\% \\
 		 & & CCC & 33.3\% & 33.8\% \\
 		 \textbf{B et al.} & 2953  &  C& 67.3\% & 68.0\% \\
 		 (2013) & & CC & 95.4\% & 85.9\% \\
 		 & & CCC & 51.3\% & 70.6\% \\
 		\bottomrule %\hline
	\end{tabular}
\end{table} 



\section{Images}
\subsection{Definition}
\par \blindtext

\subsection{Usage Advantages and Limitations}
\par \blindtext

\section{Literature review}

\par During macropathology the doctors usually devise a set of characteristics that are used as predictors for histological biopsy reference. For skin cancer pathology, \citeauthor{ac} \cite{ac} transalted  ABCD guidelines to MSI-based features and validated the relevancy of the ABCD to melanoma detection.  \citeauthor{aa} \cite{aa} evaluated the melanoma detecting system MelaFind, which analyses 10-band MSI in the visible spectrum \cite{ab}, found 9.5\% average specificity of the system compared to 3.7\% of the clinician and and 98.4\% to 78\%  average sensitivity for the task of biopsy recommendation. The positive and negative predictive values for malignant melanoma and grouped benign cases were 8.5\% to 98.8\%  and 11.7\% to 98.1\% respectively. We can see that depending on the evaluated dataset and the targeted disease, the effectiveness of detection can very greatly. 

\subsection{Images}
\par \blindtext



\subsection{Assisting Tools}
\par \blindtext  


 
\subsection{Applications in Cancer Histology} 
\par \blindtext  


\subsection{Procedure Requirements}
\par  \blindtext  
 

\subsection{Limitations of Current Research}
\par \blindtext  
