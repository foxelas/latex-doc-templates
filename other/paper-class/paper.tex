%\documentclass{intcov_report}
\documentclass[conference]{IEEEtran}
%\IEEEoverridecommandlockouts
% The preceding line is only needed to identify funding in the first footnote. If that is unneeded, please comment it out.


%%%%%%%%%%%%%%%%%%%%%%%%%%%%% Additions by foxelas  %%%%%%%%%%%%%%%%%%%%%%%%%%%%



\usepackage{graphicx} % Required for including images
\graphicspath{{img/}} % Location of the graphics files
\usepackage{comment}


\usepackage{float} % to force figure positioning

\usepackage{xcolor, soul}
\sethlcolor{yellow}


\usepackage[T1]{fontenc}
\usepackage[margin=1in]{geometry}
\usepackage{amsfonts,amsmath,amssymb}
\usepackage{textcomp} % for registered, trademark etc 

% to do list
\usepackage{enumitem}
\newlist{todolist}{itemize}{2}
\setlist[todolist]{label=$\square$}
\usepackage{pifont}
\newcommand{\cmark}{\ding{51}}%
\newcommand{\xmark}{\ding{55}}%
\newcommand{\done}{\rlap{$\square$}{\raisebox{2pt}{\large\hspace{1pt}\cmark}}%
\hspace{-2.5pt}}
\newcommand{\wontfix}{\rlap{$\square$}{\large\hspace{1pt}\xmark}}



\usepackage[parfill]{parskip}
%\usepackage{titlesec} % an interface to sectioning commands
\usepackage{enumitem}

%\usepackage{xeCJK}
%\setCJKsansfont{MS Mincho}
%\setCJKsansfont{IPAGothic}
%\setCJKmonofont{IPAGothic}

\usepackage{amsmath}


%\usepackage{underscore}

\definecolor{BrilliantRed}      {RGB}{237, 41, 57} 	    % Red  VERY-Approx PANTONE RED
\definecolor{Periwinkle}        {RGB}{136, 132, 213} 	% Periwinkle  Approximate PANTONE 2715
\definecolor{Lavender}          {RGB}{240, 146, 205} 	% Lavender  Approximate PANTONE 223
\definecolor{ForestGreen}       {RGB}{0, 105, 60} 	    % ForestGreen  Approximate PANTONE 349
\definecolor{BrickRed}          {RGB}{170, 39, 47} 	    % BrickRed  Approximate PANTONE 1805
\definecolor{NavyBlue}          {RGB}{0, 70, 173}       % NavyBlue  Approximate PANTONE 293


\usepackage{url}
\def\UrlBreaks{\do\/\do-}

\usepackage[breaklinks, colorlinks = true,
            linkcolor = Periwinkle,
            urlcolor  = Periwinkle,
            citecolor = Periwinkle,
            anchorcolor = Periwinkle]{hyperref} %produce hypertext links
            
\newcommand{\chk}[1]{{\color{BrickRed}{\textbf{Check:} #1}}}
\newcommand{\rmv}[1]{{\color{NavyBlue}{\textbf{Remove:} #1}}}
\newcommand{\rf}{{\color{ForestGreen}{\textbf{Ref}}}}
\newcommand{\reph}[1]{{\color{NavyBlue}{\textbf{Rephrase:} #1}}}
\newcommand{\dq}[1]{{\color{Lavender}{\textbf{Quote}}}\textit{ #1}}
\newcommand{\cor}[1]{{\color{BrilliantRed}{\textbf{Cor}}}\textit{ #1}}


\usepackage{orcidlink}

%%%%%%%%%%%%%%%%%%%%%%%%%%%%% End of Additions by foxelas  %%%%%%%%%%%%%%%%%%%%%%%%%%%%

\usepackage{blindtext}

\begin{document}



\title{(Tentative) A Digital Twin Framework for Traffic Monitoring in Singapore 
%*\\ 
%{\footnotesize \textsuperscript{*}Note: Sub-titles are not captured in Xplore and should not be used}
%\thanks{Identify applicable funding agency here. If none, delete this.}
}

\author{\IEEEauthorblockN{1\textsuperscript{st} Given Name Surname \orcidlink{0000-0000-0000-0000}}
\IEEEauthorblockA{\textit{dept. name of organization (of Aff.)} \\
\textit{name of organization (of Aff.)}\\
City, Country \\
email address or ORCID}
\and
\IEEEauthorblockN{2\textsuperscript{nd} Given Name Surname}
\IEEEauthorblockA{\textit{dept. name of organization (of Aff.)} \\
\textit{name of organization (of Aff.)}\\
City, Country \\
email address or ORCID}
\and
\IEEEauthorblockN{3\textsuperscript{rd} Given Name Surname}
\IEEEauthorblockA{\textit{dept. name of organization (of Aff.)} \\
\textit{name of organization (of Aff.)}\\
City, Country \\
email address or ORCID}
\and
\IEEEauthorblockN{4\textsuperscript{th} Given Name Surname}
\IEEEauthorblockA{\textit{dept. name of organization (of Aff.)} \\
\textit{name of organization (of Aff.)}\\
City, Country \\
email address or ORCID}
\and
\IEEEauthorblockN{5\textsuperscript{th} Given Name Surname}
\IEEEauthorblockA{\textit{dept. name of organization (of Aff.)} \\
\textit{name of organization (of Aff.)}\\
City, Country \\
email address or ORCID}
\and
\IEEEauthorblockN{6\textsuperscript{th} Given Name Surname}
\IEEEauthorblockA{\textit{dept. name of organization (of Aff.)} \\
\textit{name of organization (of Aff.)}\\
City, Country \\
email address or ORCID}
}


\maketitle



\begin{abstract}
\blindtext
\end{abstract}

\begin{IEEEkeywords}
component, formatting, style, styling, insert
\end{IEEEkeywords}




\section{Introduction}
\subsection{Context}
\par Citation as \cite{UserManual}.

\par \blindtext



\subsection{Goal}
\par \blindtext


\subsection{Novelty}
\par \blindtext



\subsection{Expected Outcomes}
\par \blindtext

 
\section{Methods and Materials}
\subsection{Data Sources}
\par \blindtext

\subsection{System Architecture}
\par \blindtext



\begin{figure}[h!]
\centering
\includegraphics[width=0.9\linewidth]{example-image-a}
\caption{The proposed system architecture.}
\end{figure}


\section{Results}
\par \blindtext


\begin{figure*}[h!]
\centering
\includegraphics[width=0.9\textwidth]{example-image-b}
\caption{Large figure.}
\end{figure*}


\section{Discussion}
\subsection{Future Work}
\par \blindtext

\section{Conclusion}
\par \blindtext



%\newpage
%\pagebreak
\nocite{*}
\bibliographystyle{ieeetr_fox}
\bibliography{bibliography}

\end{document}
